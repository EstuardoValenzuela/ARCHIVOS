\documentclass{article}

\usepackage{fancyhdr} % Required for custom headers
\usepackage{lastpage} % Required to determine the last page for the footer
\usepackage{extramarks} % Required for headers and footers
\usepackage[usenames,dvipsnames]{color} % Required for custom colors
\usepackage{graphicx} % Required to insert images
\usepackage{listings} % Required for insertion of code
\usepackage{courier} % Required for the courier font
\usepackage{multirow}
\usepackage{hyperref}


% Margins
\topmargin=-0.45in
\evensidemargin=0in
\oddsidemargin=0in
\textwidth=6.5in
\textheight=9.0in
\headsep=0.25in

\linespread{1.1} % Line spacing
\begin{document}
\title{Hoja de trabajo 2}
\author{Estuardo Valenzuela}
\date{26 de julio}
\maketitle
\paragraph{Ejercicio 1 \ \\ Demostrar utilizando inducci\'on}
\ \\ \[
\forall\ n.\ n^3\geq n^2
 \]
\paragraph{Soluci\'on}
\ \\ 
Aplicando propiedad de potenciacion donde:
 $a^ba^c=a^b^+^c$
\ \\
Entonces:
\begin{center}
$\rightarrow$ $n^1n^1n^1 = n^3$
\ \\ $\rightarrow$ $n^1n^1 = n^2$
\ \\ $n^3\geq n^2$
\end{center} 
\ \\
\ \\ \paragraph{Ejercicio 2 \ \\ Demostrar utilizando inducci\'on la desigualdad de Bernoulli}
\paragraph{Soluci\'on}
\ \\ Siendo x = 1 y n = 2
\ \\ entonces
\begin{center}
	$(1+1)^2 \geq 1*2$
	\ \\ $4 \geq 2$
\end{center}
\end{document}