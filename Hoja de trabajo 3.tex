\documentclass{article}

\usepackage{fancyhdr} % Required for custom headers
\usepackage{lastpage} % Required to determine the last page for the footer
\usepackage{extramarks} % Required for headers and footers
\usepackage[usenames,dvipsnames]{color} % Required for custom colors
\usepackage{graphicx} % Required to insert images
\usepackage{listings} % Required for insertion of code
\usepackage{courier} % Required for the courier font
\usepackage{multirow}
\usepackage{hyperref}
\usepackage{amsmath}
\usepackage{amssymb}

% Margins
\topmargin=-0.45in
\evensidemargin=0in
\oddsidemargin=0in
\textwidth=6.5in
\textheight=9.0in
\headsep=0.25in

\linespread{1.1} % Line spacing

%----------------------------------------------------------------------------------------
%	CODE INCLUSION CONFIGURATION
%----------------------------------------------------------------------------------------

\definecolor{MyDarkGreen}{rgb}{0.0,0.4,0.0} % This is the color used for comments
\lstloadlanguages{c} % Load Perl syntax for listings, for a list of other languages supported see: ftp://ftp.tex.ac.uk/tex-archive/macros/latex/contrib/listings/listings.pdf
\lstset{language=[sharp]c, % Use Perl in this example
	frame=single, % Single frame around code
	basicstyle=\small\ttfamily, % Use small true type font
	keywordstyle=[1]\color{Blue}\bf, % Perl functions bold and blue
	keywordstyle=[2]\color{Purple}, % Perl function arguments purple
	keywordstyle=[3]\color{Blue}\underbar, % Custom functions underlined and blue
	identifierstyle=, % Nothing special about identifiers                                         
	commentstyle=\usefont{T1}{pcr}{m}{sl}\color{MyDarkGreen}\small, % Comments small dark green courier font
	stringstyle=\color{Purple}, % Strings are purple
	showstringspaces=false, % Don't put marks in string spaces
	tabsize=5, % 5 spaces per tab
	%
	% Put standard Perl functions not included in the default language here
	morekeywords={rand},
	%
	% Put Perl function parameters here
	morekeywords=[2]{on, off, interp},
	%
	% Put user defined functions here
	morekeywords=[3]{test},
	%
	morecomment=[l][\color{Blue}]{...}, % Line continuation (...) like blue comment
	numbers=left, % Line numbers on left
	firstnumber=1, % Line numbers start with line 1
	numberstyle=\tiny\color{Blue}, % Line numbers are blue and small
	stepnumber=5 % Line numbers go in steps of 5
}

\newcommand{\horrule}[1]{\rule{\linewidth}{#1}}

% Creates a new command to include a perl script, the first parameter is the filename of the script (without .pl), the second parameter is the caption
\newcommand{\perlscript}[2]{
	\begin{itemize}
		\item[]\lstinputlisting[caption=#2,label=#1]{#1.cs}
	\end{itemize}
}

\begin{document}

\begin{tabular}{l l}
	\multirow{5}{*}{\includegraphics[width=2cm]{data/logo.png}}
 & Universidad del Istmo de Guatemala \\
 & Facultad de Ingenieria \\
 & Ing. en Telecomunicaciones \\
 & Informatica 1 \\
 & Estuardo Valenzuela
\end{tabular}
\\\\\\
\begin{center}
	\horrule{0.5pt}
	\huge{Hoja de trabajo \#3} \\
	\large{Fecha de entrega: 16 de Agosto, 2018 - 11:59pm} \\
	\horrule{1pt}
\end{center}
\section*{Ejercicio \#1 (10\%)}
Utilizando la definicion de suma ($\oplus$) para los numeros naturales unarios, llevar
a cabo la suma entre tres [$s(s(s(0)))$] y cuatro [$s(s(s(s(0))))$]. Debe elaborar todos
los pasos de forma explicita. Como referencia, se presenta nuevamente la definici\'on de
suma para numeros natruales unarios:
\[
n\oplus m := \left\{
\begin{array}{l l}
m & \mbox{si } n=o \\
n & \mbox{si } m=o \\
s(i\oplus m) & \mbox{si } n=s(i) \\
\end{array}
\right.
\]
\paragraph{Resolucion al ejercicio 1} 
\begin{center}
\ \\  $s(s(s(0)))$ $\oplus$ $s(s(s(s(0))))$
\ \\  $s(sss(0)$ $\oplus$ $sss(0)) $
\ \\  $s(s(ssss(0) \oplus ss(0)))$
\ \\  $s(s(s(ssss(0))) \oplus s(0))$
\ \\ $s(s(s(s(s(s(s(s(0))))))) \oplus 0)$
\ \\ $s(s(s(s(s(s(s(s(0))))))))$
\end{center}
\section*{Ejercicio \#2 (30\%)}
Definir inductivamente una funci\'on para multiplicar ($\otimes$) numeros naturales unarios.
{\bf Consejo: }Puede apoyarse de la definici\'on de suma estudiada durante la
clase.
\paragraph{Resoluci\'on ejercicio 2}
\ \\ $[U-U] \oplus [U \oplus U \oplus ...]$ $\rightarrow$ donde U es el primer n\'umero unitario y u es el segundo n\'umero unitario que multiplica al primero, la cantidad de U's dentro del parentesis esta limitada por el valor de u
\section*{Ejercicio \#3 (20\%)}
Verifique que su definici\'on de multiplicaci\'on es correcta multiplicando los siguientes valores:
\begin{itemize}
	\item{$s(s(s(0)))\otimes 0$}
	\ \\ $\rightarrow$$[s(s(s(0))) - s(s(s(0))))] \oplus [0]$
	\ \\ $\rightarrow [0] \oplus [0]$
	\ \\ $\rightarrow$$0$
	\item{$s(s(s(0)))\otimes s(0)$}
	\ \\ $\rightarrow [s(s(s(0)))-s(s(s(0)))] \oplus [s(s(s(0)))]$
\ \\ $\rightarrow [0] \oplus s(s(s(0)))$
\ \\ $\rightarrow [0] \oplus s((0) \oplus ss(0))$
\ \\ $\rightarrow [0] \oplus s(s(0) \oplus s(0))$
\ \\ $\rightarrow [0] \oplus s(s(s(0))) \oplus 0 $
\ \\ $\rightarrow  s(s(s(0)))$
	\item{$s(s(s(0)))\otimes s(s(0))$}
\ \\ $\rightarrow [s(s(s(0)))-s(s(s(0)))] \oplus [s(s(s(0))) \oplus s(s(0)))]$
\ \\ $\rightarrow$ $[0] \oplus [s(sss(0) \oplus s(0))]$
\ \\ $\rightarrow$ $[0] \oplus [s(s(s(sss(0))) \oplus (0))]$
\ \\ $\rightarrow$ $[0] \oplus [s(s(s(s(sss(0)))))]$
\ \\ $\rightarrow$ $[s(s(s(sss(0))))]$
\end{itemize}
\section*{Ejercicio \#4 (40\%)}
Demostrar utilizando inducci\'on:
\begin{itemize}
	\item{$a\oplus s(s(0))=s(s(a))$}
	\ \\ 
		Siendo a $=s(0)$ $\rightarrow$ $s(0) \oplus s(s(0)) = s(s(s(0)))$
		\ \\ $s(s(s(0)))=s(s(s(0)))$

	\item{$a \otimes b = b \otimes a$}
	\ \\
	 $ a > 0$ \ $b > 0 $
	 \ \\ $s(0) \otimes s(s(0)) = s(s(0)) \otimes s(0)$
	 \ \\ $[s(0)-s(0)] \oplus [s(0)\oplus s(0)] = [s(s(0)) - s(s(0))] \oplus [s(s(0))]$
	 \ \\ $[0] \oplus [s(0) \oplus s(0)] = [0] \oplus [s(s(0))] $  
	 \ \\ $s(s(0)) = s(s(0)) $

	\item{$a \otimes (b \otimes c)=(a\otimes b)\otimes c$}
	\ \\ 

		$a>0 \ b>0 \ c>0 $
		\ \\ $a=s(s(0))$
		\ \\ $b=s(0)) $
		\ \\ $c=s(s(s(0)))$
		\ \\ $\rightarrow$ $s(s(0)) \otimes ([s(0)-s(0)] \oplus [s(s(s(0)))]) = ([s(s(0))-s(s(0))] \oplus [(s(0)) \oplus s(0))] \otimes s(s(s(0)))$
		\ \\ $\rightarrow$ $s(s(0)) \otimes s(s(s(0)))=s(s(0)) \otimes s(s(s(0)))$
		\ \\ $\rightarrow$ $[s(s(0)-s(s(0)] \oplus [s(s(s(0))) \oplus s(s(s(0)))] = [s(s(0)) -s(s(0))] \oplus [s(s(s(0))) \oplus s(s(s(0)))]$
		\ \\ $\rightarrow$ $ssssss(0) = ssssss(0)$

	\item{$(a\oplus b)\otimes c = (a\otimes b) \oplus (b \otimes c)$}
	\ \\ $a=0 \\ b=s(0) \\ c=s(s(0))$
		\ \\ $\rightarrow$ $ (a\oplus b)\otimes c = (a\otimes b) \oplus (b \otimes c)$
		\ \\ $\rightarrow$ $(0+s(0))\otimes s(s(0))=(0\otimes s(0)) \oplus (s(0) \otimes s(s(0))$
		\ \\ $\rightarrow$ $s(0) \otimes s(s(0)) = [s(0)-s(0)] \oplus[0] \oplus [s(s(0))-s(s(0))] \oplus [s(s(0))]$
		\ \\ $\rightarrow$ $s(s(0)) = 0 + 0 + s(s(0))$
		\ \\ $\rightarrow$ $s(s(0)) = s(s(0))$
		\ \\ $\rightarrow$ $ss(0)=ss(0)$
\end{itemize}

\end{document}