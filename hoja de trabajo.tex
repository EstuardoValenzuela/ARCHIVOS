\documentclass{article}

\usepackage{fancyhdr} % Required for custom headers
\usepackage{lastpage} % Required to determine the last page for the footer
\usepackage{extramarks} % Required for headers and footers
\usepackage[usenames,dvipsnames]{color} % Required for custom colors
\usepackage{graphicx} % Required to insert images
\usepackage{listings} % Required for insertion of code
\usepackage{courier} % Required for the courier font
\usepackage{multirow}
\usepackage{hyperref}


% Margins
\topmargin=-0.45in
\evensidemargin=0in
\oddsidemargin=0in
\textwidth=6.5in
\textheight=9.0in
\headsep=0.25in

\linespread{1.1} % Line spacing
\begin{document}
\title{Hoja de trabajo 2}
\author{Estuardo Valenzuela}
\date{26 de julio}
\maketitle
\paragraph{Ejercicio 1 \ \\ Demostrar utilizando induccion}
\ \\ \[
\forall\ n.\ n^3\geq n^2
 \]
\paragraph{Solucion}
\ \\ 
Aplicando propiedad de potenciacion donde:
\ \\ $a^b×a^c=a^b^+^c$
\ \\
Entonces:
\begin{center}
$\rightarrow$ $n^1×n^1×n^1 = n^3$
\ \\ $\rightarrow$ $n^1×n^1 = n^2$
\ \\ $n^3\geq n^2$
\end{center} 
\ \\
\ \\ \paragraph{Ejercicio 3 \ \\ A}
\ \\ 1,1 1,2 1,3 1,4 1,5 1,6
\ \\ 2,1 2,2 2,3 2,4 2,5 2,6
\ \\ 3,1 3,2 3,3 3,4 3,5 3,6
\ \\ 4,1 4,2 4,3 4,4 4,5 4,6
\ \\ 5,1 5,2 5,3 5,4 5,5 5,6
\ \\ 6,1 6,2 6,3 6,4 6,5 6,6
\ \\ \paragraph{B}
\ \\ N1,N2 $\rightarrow$  Donde N1 es dado 1 y N2 es dado 2
\ \\ \paragraph{C}
\ \\ Nos podemos asegurar que el resultado que caiga a la hora de lanzar los dados sera uno del inciso A, ya que son las unicas combinaciones posibles, ademas el lanzamiento no se pude predecir ya que siempre es al azar 
\end{document}